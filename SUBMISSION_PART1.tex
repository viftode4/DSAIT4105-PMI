\documentclass[11pt,a4paper]{article}

% Packages
\usepackage[utf8]{inputenc}
\usepackage[T1]{fontenc}
\usepackage{amsmath,amssymb,amsfonts}
\usepackage{graphicx}
\usepackage{booktabs}
\usepackage{array}
\usepackage{listings}
\usepackage{xcolor}
\usepackage{hyperref}
\usepackage[margin=1in]{geometry}
\usepackage{fancyhdr}
\usepackage{titlesec}

% Code listing style
\definecolor{codegreen}{rgb}{0,0.6,0}
\definecolor{codegray}{rgb}{0.5,0.5,0.5}
\definecolor{codepurple}{rgb}{0.58,0,0.82}
\definecolor{backcolour}{rgb}{0.95,0.95,0.92}

\lstdefinestyle{mystyle}{
    backgroundcolor=\color{backcolour},
    commentstyle=\color{codegreen},
    keywordstyle=\color{magenta},
    numberstyle=\tiny\color{codegray},
    stringstyle=\color{codepurple},
    basicstyle=\ttfamily\footnotesize,
    breakatwhitespace=false,
    breaklines=true,
    captionpos=b,
    keepspaces=true,
    numbers=left,
    numbersep=5pt,
    showspaces=false,
    showstringspaces=false,
    showtabs=false,
    tabsize=2,
    frame=single
}
\lstset{style=mystyle}

% Header/Footer
\pagestyle{fancy}
\fancyhf{}
\rhead{DSAIT4105 - Probabilistic Models and Inference}
\lhead{Part I: Problem Library}
\rfoot{Page \thepage}

% Title
\title{\textbf{Probabilistic Models and Inference}\\[0.5em]
\Large Part I: Problem Library Submission}
\author{[Your Name]}
\date{\today}

\begin{document}

\maketitle

\begin{abstract}
This submission presents two simulation-based inference problems designed to stress test standard inference procedures. The models focus on physics domains: (1) a \textbf{Projectile/Artillery Model} demonstrating multimodality and parameter correlation challenges, and (2) a \textbf{Double Pendulum Model} demonstrating chaotic dynamics that break gradient-based inference. Both models are implemented in Gen.jl with full probabilistic programs and theoretical analysis of inference procedure performance.
\end{abstract}

\tableofcontents
\newpage

%==============================================================================
\section{Model 1: Projectile/Artillery Trajectory Inference}
%==============================================================================

\subsection{Problem Statement}

Given noisy observations of where a projectile lands, infer the initial velocity $v_0$ and launch angle $\theta$ used to fire it. This is a classic inverse problem in ballistics with applications in artillery targeting, sports analytics, and forensic trajectory reconstruction.

\subsection{Simulator Description}

The simulator implements 2D ballistic motion under gravity. The trajectory equations are:

\begin{align}
x(t) &= x_0 + v_0 \cos(\theta) \cdot t \\
y(t) &= y_0 + v_0 \sin(\theta) \cdot t - \frac{1}{2}gt^2
\end{align}

The impact position (where $y = 0$) can be computed analytically:
\begin{equation}
x_{\text{impact}} = \frac{v_0^2 \sin(2\theta)}{g}
\end{equation}

\textbf{Inputs:}
\begin{itemize}
    \item $v_0$: Initial velocity (m/s)
    \item $\theta$: Launch angle (radians)
\end{itemize}

\textbf{Output:}
\begin{itemize}
    \item Impact position (x-coordinate where $y = 0$)
\end{itemize}

The simulator executes in milliseconds, well under the recommended few-seconds threshold.

\subsection{Probabilistic Model}

\begin{lstlisting}[language=Python, caption={Gen.jl probabilistic model for projectile inference}]
@gen function projectile_model(observed_impacts, noise_std)
    # Priors
    v0 = ({:v0} ~ uniform(10.0, 50.0))      # Initial velocity (m/s)
    theta = ({:theta} ~ uniform(pi/18, 4*pi/9)) # Angle: 10 to 80 degrees

    # Run simulator
    x_impact = simulate_projectile(v0, theta)

    # Likelihood: observed impacts are noisy measurements
    for (i, obs) in enumerate(observed_impacts)
        {(:impact, i)} ~ normal(x_impact, noise_std)
    end

    return (v0, theta, x_impact)
end
\end{lstlisting}

\subsection{Why This Model Tests Inference}

\subsubsection{Challenge 1: Multimodal Posterior}

The same target distance can be achieved with two different $(v_0, \theta)$ combinations. From the range equation $R = \frac{v_0^2 \sin(2\theta)}{g}$, we see that $\sin(2\theta) = \sin(2(\frac{\pi}{2} - \theta))$, meaning complementary angles achieve the same range:

\begin{table}[h]
\centering
\begin{tabular}{lccc}
\toprule
\textbf{Solution Type} & \textbf{Angle} & \textbf{Velocity} & \textbf{Trajectory} \\
\midrule
Low Arc & $\sim 25°$ & $\sim 35$ m/s & Flat, fast \\
High Arc & $\sim 65°$ & $\sim 35$ m/s & Tall, slow \\
\bottomrule
\end{tabular}
\caption{Two solutions achieving the same impact distance}
\end{table}

This creates a \textbf{bimodal posterior} that sampling methods must explore. Methods that get stuck in one mode will underestimate posterior uncertainty.

\subsubsection{Challenge 2: Parameter Correlation}

Velocity and angle trade off: increasing angle requires adjusting velocity to maintain range. Along iso-range curves, we have:
\begin{equation}
v_0 = \sqrt{\frac{Rg}{\sin(2\theta)}}
\end{equation}

This creates elongated, curved posterior regions that are difficult for methods with axis-aligned proposals.

\subsection{Inference Procedure Analysis}

\begin{table}[h]
\centering
\begin{tabular}{p{3.5cm}p{2cm}p{8cm}}
\toprule
\textbf{Method} & \textbf{Performance} & \textbf{Explanation} \\
\midrule
Importance Sampling & POOR & Samples from prior rarely land in either mode. High variance, weight degeneracy. ESS $\ll$ N. \\
\midrule
Metropolis-Hastings & MODERATE & Can explore within modes but struggles to jump between them. Correlation slows mixing. \\
\midrule
HMC & GOOD & Smooth likelihood allows gradient-based exploration. Efficiently navigates curved posterior geometry. \\
\midrule
Particle Filtering & N/A & Problem is not sequential in time. \\
\bottomrule
\end{tabular}
\caption{Theoretical analysis of inference procedures for the projectile model}
\end{table}

\subsection{Code Location}
\texttt{models/projectile\_artillery.jl}

%==============================================================================
\section{Model 2: Double Pendulum Parameter Inference}
%==============================================================================

\subsection{Problem Statement}

Given noisy observations of a double pendulum's trajectory, infer the physical parameters (lengths, masses) and initial conditions. The double pendulum is a canonical example of a chaotic dynamical system.

\subsection{Simulator Description}

The simulator solves the double pendulum equations of motion using RK4 integration.

\textbf{State:} $[\theta_1, \theta_2, \omega_1, \omega_2]$ (angles and angular velocities)

\textbf{Dynamics:} Derived from Lagrangian mechanics, the angular accelerations are:
\begin{align}
\alpha_1 &= \frac{-g(2m_1 + m_2)\sin\theta_1 - m_2 g\sin(\theta_1 - 2\theta_2) - 2\sin(\Delta\theta)m_2(\omega_2^2 L_2 + \omega_1^2 L_1\cos(\Delta\theta))}{L_1(2m_1 + m_2 - m_2\cos(2\Delta\theta))} \\
\alpha_2 &= \frac{2\sin(\Delta\theta)(\omega_1^2 L_1(m_1 + m_2) + g(m_1 + m_2)\cos\theta_1 + \omega_2^2 L_2 m_2\cos(\Delta\theta))}{L_2(2m_1 + m_2 - m_2\cos(2\Delta\theta))}
\end{align}
where $\Delta\theta = \theta_1 - \theta_2$.

\textbf{Inputs:}
\begin{itemize}
    \item Physical parameters: $L_1, L_2$ (lengths), $m_1, m_2$ (masses)
    \item Initial conditions: $\theta_1, \theta_2, \omega_1, \omega_2$
\end{itemize}

\textbf{Output:}
\begin{itemize}
    \item Trajectory of $(x, y)$ positions for both pendulum bobs
\end{itemize}

\subsection{Probabilistic Model}

\begin{lstlisting}[language=Python, caption={Gen.jl probabilistic model for double pendulum inference}]
@gen function double_pendulum_model(observed_positions, obs_times, noise_std)
    # Priors on parameters
    L1 = ({:L1} ~ uniform(0.5, 1.5))
    L2 = ({:L2} ~ uniform(0.5, 1.5))
    m2 = ({:m2} ~ uniform(0.5, 2.0))

    # Priors on initial conditions
    theta1_init = ({:theta1_init} ~ uniform(-pi, pi))
    theta2_init = ({:theta2_init} ~ uniform(-pi, pi))
    omega1_init = ({:omega1_init} ~ normal(0, 0.5))
    omega2_init = ({:omega2_init} ~ normal(0, 0.5))

    # Run simulator
    trajectory = simulate_double_pendulum(params, initial_state)

    # Likelihood at observation times
    for (i, t) in enumerate(obs_times)
        sim_x, sim_y = trajectory_at_time(trajectory, t)
        {(:obs_x, i)} ~ normal(sim_x, noise_std)
        {(:obs_y, i)} ~ normal(sim_y, noise_std)
    end
end
\end{lstlisting}

\subsection{Why This Model Tests Inference}

\subsubsection{Challenge 1: Chaotic Dynamics}

The double pendulum exhibits \textbf{deterministic chaos}:
\begin{itemize}
    \item Exponential divergence of nearby trajectories
    \item Positive Lyapunov exponent
    \item A $0.0001$ radian perturbation leads to $>1$ meter divergence after 5 seconds
\end{itemize}

\textbf{Experimental demonstration:}
\begin{table}[h]
\centering
\begin{tabular}{ccc}
\toprule
\textbf{Time (s)} & \textbf{Divergence (m)} & \textbf{Amplification} \\
\midrule
1.0 & 0.0002 & $2\times$ \\
2.0 & 0.0013 & $13\times$ \\
3.0 & 0.1872 & $1,872\times$ \\
5.0 & 1.1119 & $11,119\times$ \\
7.0 & 2.9835 & $29,835\times$ \\
\bottomrule
\end{tabular}
\caption{Trajectory divergence from $10^{-4}$ radian perturbation}
\end{table}

\subsubsection{Challenge 2: Gradient Catastrophe}

Gradients $\partial(\text{output})/\partial(\text{parameters})$ are meaningless in the chaotic regime:
\begin{itemize}
    \item Finite differences don't converge
    \item Sign flips unpredictably
    \item Magnitude explodes exponentially
\end{itemize}

\begin{table}[h]
\centering
\begin{tabular}{cc}
\toprule
$\varepsilon$ & $\partial x_{\text{final}} / \partial \theta_{1,\text{init}}$ \\
\midrule
0.1 & 8.76 \\
0.01 & 46.68 \\
0.001 & 583.96 \\
0.0001 & $-4096.76$ \\
\bottomrule
\end{tabular}
\caption{Finite difference gradients do NOT converge as $\varepsilon \to 0$}
\end{table}

Gradients fluctuate in both sign and magnitude. This breaks all gradient-based methods including HMC.

\subsubsection{Challenge 3: Fractal Likelihood Surface}

The parameter regions yielding ``good'' trajectories form complex, disconnected sets with fractal-like boundaries. Local search cannot navigate this landscape effectively.

\subsection{Inference Procedure Analysis}

\begin{table}[h]
\centering
\begin{tabular}{p{3.5cm}p{2.5cm}p{7.5cm}}
\toprule
\textbf{Method} & \textbf{Performance} & \textbf{Explanation} \\
\midrule
Importance Sampling & VERY POOR & Prior samples essentially never match observations. All weights on negligible fraction of samples. \\
\midrule
Metropolis-Hastings & POOR & Likelihood landscape so rough that most moves rejected. Cannot navigate to good regions. \\
\midrule
HMC & FAILS & Gradients are wrong, not just noisy. Leapfrog integrator diverges. Near-zero acceptance rate. \\
\midrule
Particle Filtering & POOR & Particles diverge rapidly due to chaos. Collapses to single particle quickly. \\
\bottomrule
\end{tabular}
\caption{Theoretical analysis of inference procedures for the double pendulum model}
\end{table}

\subsection{This is an Adversarial Test Case}

The double pendulum demonstrates a \textbf{fundamental limitation}: some probabilistic models with valid likelihoods are \textbf{computationally intractable} for standard inference. No standard algorithm can efficiently explore this posterior without problem-specific modifications.

\subsection{Code Location}
\texttt{models/double\_pendulum.jl}

%==============================================================================
\section{Comparison Summary}
%==============================================================================

\begin{table}[h]
\centering
\begin{tabular}{lcc}
\toprule
\textbf{Aspect} & \textbf{Projectile} & \textbf{Double Pendulum} \\
\midrule
Domain & Games/Physics & Physics \\
Simulator Type & Closed-form ballistics & ODE integration \\
Primary Challenge & Multimodality & Chaos \\
Secondary Challenge & Correlation & Gradient catastrophe \\
Likelihood Surface & Smooth, multimodal & Rough, fractal \\
Gradients Useful? & Yes & No \\
Best Standard Method & HMC & None (all fail) \\
Difficulty & Moderate & Extreme \\
\bottomrule
\end{tabular}
\caption{Comparison of the two models}
\end{table}

\subsection{All Four Inference Procedures}

\begin{table}[h]
\centering
\begin{tabular}{p{3cm}p{5.5cm}p{5.5cm}}
\toprule
\textbf{Procedure} & \textbf{Projectile Model} & \textbf{Double Pendulum Model} \\
\midrule
Importance Sampling & POOR - multimodal posterior causes weight degeneracy & VERY POOR - chaos makes matching nearly impossible \\
\midrule
Metropolis-Hastings & MODERATE - correlation slows mixing between modes & POOR - rough likelihood causes very low acceptance \\
\midrule
HMC & GOOD - smooth gradients enable efficient exploration & FAILS - gradients are meaningless in chaotic regime \\
\midrule
Particle Filtering & N/A - not a sequential problem & POOR - chaotic divergence causes rapid particle collapse \\
\bottomrule
\end{tabular}
\caption{Summary of inference procedure performance on both models}
\end{table}

%==============================================================================
\section{Conclusions}
%==============================================================================

\subsection{What We Learned}

\begin{enumerate}
    \item \textbf{No universal inference algorithm exists.} Each model has characteristics that favor or break different methods.

    \item \textbf{Simulator properties matter.} Smoothness, chaos, and gradient behavior determine which algorithms can work.

    \item \textbf{Multimodality is challenging but manageable.} With methods like HMC or tempered MCMC, multimodal posteriors can be explored.

    \item \textbf{Chaos is fundamentally difficult.} Standard algorithms all fail on chaotic systems. This requires specialized approaches such as ABC, summary statistics, or short-window inference.
\end{enumerate}

\subsection{For Part II}

Custom inference strategies to explore:

\textbf{Projectile Model:}
\begin{itemize}
    \item Custom proposals that respect $v_0$-$\theta$ correlation
    \item Parallel tempering to jump between modes
    \item Reparameterization to (range, max\_height) instead of $(v_0, \theta)$
\end{itemize}

\textbf{Double Pendulum Model:}
\begin{itemize}
    \item ABC with trajectory summary statistics
    \item Short-window inference before chaos dominates
    \item Likelihood tempering with annealing schedules
\end{itemize}

%==============================================================================
\section{Files Included}
%==============================================================================

\begin{verbatim}
DSAIT4105-PMI/
|-- models/
|   |-- projectile_artillery.jl      # Model 1: Projectile/Artillery
|   |-- double_pendulum.jl           # Model 2: Double Pendulum
|-- test_projectile.jl               # Test script for Model 1
|-- test_double_pendulum.jl          # Test script for Model 2
|-- SUBMISSION_PART1.tex             # This document
\end{verbatim}

\textbf{To run tests:}
\begin{verbatim}
julia test_projectile.jl
julia test_double_pendulum.jl
\end{verbatim}

%==============================================================================
\section*{References}
%==============================================================================

\begin{enumerate}
    \item Gen.jl Documentation: \url{https://www.gen.dev/}
    \item Strogatz, S. (2015). \textit{Nonlinear Dynamics and Chaos}. Westview Press.
    \item MacKay, D. (2003). \textit{Information Theory, Inference, and Learning Algorithms}. Cambridge University Press.
    \item Bishop, C. (2006). \textit{Pattern Recognition and Machine Learning}. Springer.
\end{enumerate}

\end{document}
